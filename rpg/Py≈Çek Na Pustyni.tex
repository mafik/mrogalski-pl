% Created 2010-06-02 śro 23:16
\documentclass[11pt]{article}
\usepackage[utf8]{inputenc}
\usepackage[T1]{fontenc}
\usepackage{graphicx}
\usepackage{longtable}
\usepackage{hyperref}


\title{Pyłek Na Pustyni}
\author{MG: Mafik}
\date{środa, 2 czerwca 2010}

\begin{document}

\maketitle


\section*{Wstęp}
\label{sec-1}

  Yfez Kahraman, potężny czarnoksiężnik, wezyr króla Arabii, od
  wieków spoczywa pod piaskami pustyni, w piramidzie na zachodzie
  Arabii. Razem z nim spoczywają tam ogromne skarby.

  Uczestniczycie w wyprawie zorganizowanej przez jednego z ojców
  Talabeckiego rodu Kellerów, Wilfreda Kellera, oraz jego
  przyjaciela, szejka z Arabii, Sonera Sari. Celem wyprowy jest
  zdobycie kosztowności spoczywających pod piaskami i
  odrtansportowanie ich do Imperium.

  Po dwóch miesiącach podróży w nadzwyczaj sprzyjających warunkach,
  dotarliście do piramidy, grobowca Yfeza. Gigantyczna budowla. Swym
  rozmiarem przewyższa wszystko co widzieliście w
  Imperium. Obozujecie w oazie o pół dnia drogi od samego
  grobowca. Chłodnie noce i upał dnia zdają się zupełnie nie
  przeszkadzać w obozowaniu i podróży.

  Dziś jest środa, planowo data zejścia do grobowca. W
  poniedziałkową noc zdarzyło się jednak coś co przerwało dobrą
  passę i może oddalić zejście do pod ziemię. Rankiem, Wilfred
  Keller, główny fundator wyprawy nie obudził się. W istocie, był
  martwy. Jego ciało nie posiadało żadnych widocznych obrażeń, a
  jedynie sińce pod oczami sugerowały że mógł zostać
  otruty. Pomiędzy członkami wyprawy zapanowało poruszenie. Ciało
  Wilfreda z uwagi na upał musiało zostać jak najszybciej
  pochowane. Zachowując pamiątki do przywiezienia z powrotem do
  domu, pochowaliście go na skrawku zdrowej ziemi na skraju oazy.

  Na tym zła passa się nie skończyła, bowiem następnego dnia, Soner
  Sari, jedyny potencjalny konkurent śp. Wilfreda udał się na kąpiel
  i już z niej nie wrócił. Zniknął pozostawiając po sobie ubrania i
  cały swój dobytek. Również i jego pochowaliście na skraju oazy.

  Dziś jest pogodny, o ile nie liczyć pochmurnych nastrojów,
  ranek. Zbliża się chwila zejścia do grobowca. Czy jednak będzie to
  bezpieczne? Wśród was może być morderca.

\section*{Dżiny}
\label{sec-2}

  Dżiny to złe i potężne stworzenia. Dawniej, najpotężniejsze z nich
  rządziły światem jednak rodzaj potężnych czarnoksiężników, chodzących
  po ziemi na długo przed pojawieniem się ludzi, zniszczył niemal
  wszystkie. Tylko największe przetrwały i po odejściu
  czanoksiężników, powróciły.

  Dżin to duch o niesamowitej mocy. Potrafią przybierać materialne
  postaci, a sama ich obecność wpływa na przyrodę dookoła. Z tego
  powodu wokół domostw największych dżinów powstawały pustynie i
  bagna. Największe z nich zamieszkują środek Sahary, wielkiej
  pustynii na zachodzie Arabii. Wywołują tam permanentne susze,
  burze piaskowe, a nieostrożnych podróżnych wciągają w ruchome
  piaski.

  Tysiące lat temu żył w Arabii czarnoksiężnik, wezyr króla
  Laokira. Nazywał się Yfez Kahraman Posiadł on umiejętność
  zaklinania dżinów tak by spełniały 
  jego polecenia. Dżiny zaklinał w naczyniach takich jak dzbany,
  kosze, metalowe pudełeczka, czy nawet w zbrojach. 

  Z ich pomocą, Kahraman dokonał zamachu na swego króla, oraz szybko
  podporządkował sobie całą Arabię. Dżiny strzegły go i walczyły za
  niego.

  Yfez spoczął w grobowcu, pod piramidą na wschodzie Arabii. Razem z
  nim, w grobowcu zakopano magiczną lampę. W niej zaklęty został
  najpotężniejszy z dżinów, o imieniu Baltazar. To niszczycielski
  dżin pustyni który spełni każde życzenie tego kto posiądzie lampę.

\section*{Postaci Graczy}
\label{sec-3}

\subsection*{Hakim Alzadie'}
\label{sec-3.1}


\begin{center}
\begin{tabular}{ll}
 Gracz  &  Wynik  \\
\hline
 Tomek  &  1/3    \\
\end{tabular}
\end{center}


\subsubsection*{Prywatny opis}
\label{sec-3.1.1}


\begin{center}
\begin{tabular}{rrrrr}
 Moc  &  Gibkość  &  Walka  &  Mądrość  &  Siła woli  \\
\hline
   2  &        3  &      3  &        2  &          2  \\
\end{tabular}
\end{center}


    W wyprawie bierzesz udział jako dobry towarzysz szejka Sonera
    Sari. Pełnisz tu obowiązki przewodnika. Znasz ten region jako że
    wielokrotnie tędy podróżowałeś i jesteś w stanie bezpiecznie
    przeprowadzić karawanę. Przewodnikiem jesteś jednak tylko
    oficjalnie. Mniej oficjalnie, jesteś Assasinem. Płatnym mordercą, 
    od dziecka szkolonym w klasztorze. Soner Sari w istocie był
    efendi, starszym, zakonu i prowadził interesy na zewnątrz. Soner
    Sari był również przyjacielem Wilfreda, fundatora wyprawy. Oboje
    jednak zginęli.

    Kodeks nakazuje ci podjęcie misji którą przerwał Soner. Problem
    polega na tym że niewiele ci o niej wiadomo. Soner mówił wiele
    jednak jako że to nie wy mieliście być tu zwierzyną, nie zrobił z
    ciebie oficjalnego sekundanta. Wiesz jedno. Skarby które znajdują
    się w grobowcu są nic nie warte w porównaniu z jednym, magicznym
    przedmiotem. Magiczna Lampa, która jest tam ukryta stanowi dom
    potężnego dżina pustyni. Każdy kto ją posiądzie, posiądzie władzę
    nad nim.
\subsubsection*{Publiczny opis}
\label{sec-3.1.2}

    Młody Arab, w wieku ok 25-29 lat. Towarzysz Sonera Sari i zarazem
    wasz przewodnik przez ten ciężki teren. Wysportowany i schludny
    (chyba żaden Arab poza nim się nie goli). Nosi się w luźnych
    czarnych ubraniach i zapewnia karawanie wspaniałe wsparcie od
    strony logstycznej. Zajmuje się polowaniem, gotowaniem, oraz
    kieruje rozbijaniem obozowiska. Mówi jedynie po Arabsku.

    Jego ulubiony temat rozmów to\ldots{}
\subsubsection*{Cele}
\label{sec-3.1.3}

\begin{itemize}
\item dowiedzieć się jak najwięcej o lampie
\item zdobyć lampę
\item dowiedzieć się kto zabił Sonera Sari
\end{itemize}
\subsubsection*{Dodatek: Kodeks Assasinów}
\label{sec-3.1.4}

\begin{enumerate}
\item Strzec tajemnicy i dobra zakonu
\item Być posłusznym względem zwierzchników
\item Za wszelką cenę wykonywać zlecone zadanie
\end{enumerate}
\subsubsection*{Atuty}
\label{sec-3.1.5}

\begin{itemize}
\item udajesz że znasz tylko Arabski

      Przez to pozostali prowadzą przy tobie poufne rozmowy. Czasami
      wiedza może dać ogromną przewagę. Minus jest taki że do
      komunikacji z Malkiorem zawsze korzystasz z pomocy tłumaczki, Ady.
\item wszyscy mają cię za Araba, nie wiedzą że jesteś Assasinem
\item jesteś przewodnikiem drużyny na pustyni
\item specjalne zdolności

\begin{itemize}
\item 3 x orzech Tijan'e

        Rzuca się nim o ziemię. Orzech wytwarza obłok dymu i teleportuje
        cię w pobliskie miejsce nad którym się skupisz (do stu metrów,
        im dalej tym trudniej)
\item strefa cienia

        Połknięcie orzecha sprawia że przez następne kilka minut
        wydychasz gęsty, czarny dym który błyskawicznie pokrywa odszar
        o promieniu do 5m i całkowicie blokuje na nim widoczność
\item rozmawianie z wężami

        Potrafisz rozmawiać z wężami. Kiedy nie stwarza to dla nich
        zagrożenia, wykonują twoje polecenia i odpowiadają na pytania.
\item doskonały zmysł przestrzeni, audiolokacja

        Gdy raz spojrzysz na pomieszczenie jesteś w stanie bezbłędnie
        poruszać się w nim nawet z zamkniętymi oczami. Potrafisz
        również nadzwyczaj dokładnie określić miejsce z którego dobył
        się dany odgłos. Obie zdolności posiadłeś dzięki treningowi
        sztuk walki w klasztorze Assasinów.
\end{itemize}

\end{itemize}
\subsection*{Malkior Keller}
\label{sec-3.2}


\begin{center}
\begin{tabular}{ll}
 Gracz  &  Wynik  \\
\hline
 Aldik  &  2/2    \\
\end{tabular}
\end{center}


\subsubsection*{Prywatny opis}
\label{sec-3.2.1}


\begin{center}
\begin{tabular}{rrrrr}
 Moc  &  Czujność  &  Walka  &  Mądrość  &  Siła woli  \\
\hline
   3  &         2  &      3  &        2  &          3  \\
\end{tabular}
\end{center}


    Jako były dowódca regimentu armii wiesz jak obchodzić się z
    ludźmi.
    
    Jesteś bratankiem śp. Wilfreda. Właściwie byłeś. Dopóki ktoś go
    nie zamordował. Coś jest wyraźnie na rzeczy bowiem dzień
    wcześniej Wilfred powiedział ci że wybieracie się do grobowca nie
    tylko po skarby, że wiecie o czymś jeszcze, o czym inni nie mają
    pojęcia. Zdążył jedynie powiedzieć ci że chciałby z Tobą o tym
    porozmawiać, oraz że to długi temat, dlatego zarezerwował na
    niego cały dzień. Tego dnia, Wilfred się jednak nie obudził.

    Soner Sari, który mógłby być jedynym konkurentem do podziemnego
    skarbu zniknął jednak dzień później. Wilfred był jednak
    jego serdecznym przyjacielem. Czy Soner mógł zabić własnego
    przyjaciela? No i kto zabił Sonera?

    Twoim osobistym znajomym jest także Kurt, młody czarodziej,
    praktykujący kiedyś jako specjalista od magii w twoim
    regimencie. Być może będzie w stanie pomóc ci w dojściu, co się
    stało.
\subsubsection*{Publiczny opis}
\label{sec-3.2.2}

    Bratanek śp. Wilfreda. Potężnie zbudowany mężczyzna, o głębokim
    głosie i budzącym respokt uzbrojeniu. Niewielu śmiałków wyrusza
    na pustynię w ozdobnej zbroi oraz mieczem półtoraręcznym. Z
    natury spokojny, sprawia wrażenie niezwykle roztropnego. Obecnie
    dowódca wyprawy. Jedyny problem to że nie zna arabskiego, przez
    co z Hakimem musi porozumiewać sie poprzez Adę.

    Jego ulubiony temat rozmów to\ldots{}
\subsubsection*{Cele}
\label{sec-3.2.3}

\begin{itemize}
\item poznać prawdziwy motyw Wilfreda
\item dowiedzieć się kto kogo zabił
\end{itemize}
\subsubsection*{Atuty}
\label{sec-3.2.4}

\begin{itemize}
\item udajesz że nie znasz arabskiego
      Tak naprawdę to znasz. W ten sposób jesteś jednak w stanie
      słuchać rozmów reszty wyprawy kiedy nie wiedzą że cokolwiek
      rozumiesz.
\item dowódca wyprawy
\item wyśmienity żołnierz
\item znajomy czarodziej (Kurt)
\end{itemize}
\subsection*{Ada Meier}
\label{sec-3.3}


\begin{center}
\begin{tabular}{ll}
 Gracz  &  Wynik  \\
\hline
 Ania   &  1/4    \\
\end{tabular}
\end{center}


\subsubsection*{Prywatny opis}
\label{sec-3.3.1}


\begin{center}
\begin{tabular}{rrrr}
 Czujność  &  Magia  &  Mądrość  &  Siła woli  \\
\hline
        3  &      3  &        3  &          3  \\
\end{tabular}
\end{center}


    Podróżujesz jako magiczna asysta karawany, oraz przewodniczka po
    zejściu do grobowca. Nie poszukejesz skarbów, jednak wiesz że pod
    piramidą znajduje się ukryta komora z magiczną lampą. W lampie
    drzemie dżin który wykona wszystkie polecenia posiadacza
    lampy. Dżin to w istocie demon Khorna, boga mordu. Wiesz że
    budzenie go ściągnie na was co najwyżej kłopoty. Możliwe że ktoś
    przypadkiem odnajdzie lampę. Zadbaj o to by w takim wypadku nie
    otwarzył jej wieczka.

    Wejście do grobowca zostało zamknięte jednorazowo, bez możliwości
    otworzenia go. Znalazłaś jednak sposób na zejście na dół. Otóż
    posiadasz fiolkę z atramentem, z czasów życia Yfeza. Zaklęty w
    niej demon po uwolnieniu ucieknie do swojego pana. Aby przedostać
    się przez pieczęć będzie musiał ją zniszczyć i tym samym otworzyć
    ją dla was.

    Jesteś w stanie otworzyć grobowiec, lecz nie poszukujesz jego
    skarbów. Jedynie wspomagasz w tym resztę wyprawy. Co tak naprawdę
    tobą kieruje? Otóż podróżuje z tobą para Assasinów, płatnych
    morderców od dziecka szkolonych w sztukach zabijania. Tak
    naprawdę nie jesteś czarodziejką kolegium złota, lecz mistrynią
    magii kolegium cienia. Od dwóch lat podajesz się za Adę Meier i
    pracujesz w kolegium złota szpiegując przy tym ich
    działania. Magowie cienia od dawna poszukują sekretu Assasinów,
    tajemnej formuły która sprawia że mutują oni zyskując
    nadnaturalne zdolności. Twoje zadanie to przekonać (kłamstwem
    będź nie) Hakima Alzadie' do udania się do imperium, do
    podniszczałej kamienicy, w której mieści się budynek Kolegium
    Cienia. Dobrze by było jakby Hakim dotarł żywy jednak nie jest to
    absolutnym wymogiem.
\subsubsection*{Publiczny opis}
\label{sec-3.3.2}

    Ada jest młodą, jasnowłosą czarodziejką. Ma szaro-zielone oczy,
    krótkie, lekko potargane włosy za którymi skryty jest diadem z
    diamentem. Nosi skórzaną kamizelkę, białą koszulę i proste
    spodnie. Spod kamizelki dobywa się często grzechot metalowych
    przedmiotów.
    
    Jest to osoba cicha i spokojna. Sprawia wrażenie czujnej i
    ostrożnej. Jej ulubiony temat rozmów to\ldots{}
\subsubsection*{Cele}
\label{sec-3.3.3}

\begin{itemize}
\item kto kogo zabił?
\item chronić tajemnicy Kolegium Cienia
\item schwytać Assasina
\item nie dopuścić do wypuszczenia dżina
\end{itemize}
\subsubsection*{Atuty}
\label{sec-3.3.4}

\begin{itemize}
\item tłumaczka
\item przewodnik pod ziemią
\item udawany mag Kolegium Złota
\item ukryta mistrzyni magii Kolegium Cienia
\item wiesz o magicznym przedmiocie
\end{itemize}
\subsection*{Kurt Bohlmann}
\label{sec-3.4}


\begin{center}
\begin{tabular}{ll}
 Gracz  &  Wynik  \\
\hline
 Qba    &  2/3    \\
\end{tabular}
\end{center}


\subsubsection*{Prywatny opis}
\label{sec-3.4.1}


\begin{center}
\begin{tabular}{rrrrr}
 Moc  &  Czujność  &  Magia  &  Mądrość  &  Siła woli  \\
\hline
   2  &         3  &      1  &        2  &          3  \\
\end{tabular}
\end{center}


    Jesteś terminującym uczniem kolegium niebios. Od ok. miesiąca
    pracujesz jako asystent wędrownej czarodziejki Kolegium Złota,
    Ady Meier. Kiedy Ada zdecydowała że wyrusza na wyprawę do Arabii
    stwierdziłeś że nie możesz przegapić takiej okazji i ruszyłeś
    razem z nią.

    Podróżuje z wami Malkior Keller. Jakiś rok temu, kiedy odbywałeś
    praktyki na asyście magicznej w oddziale wojska służyłeś właśnie
    pod jego dowództwem. Widocznie Malkior ma pamięć do twarzy bo
    dobrze cię zapamiętał i ze wszystkich członków wyprawy
    najbardziej ufa właśnie tobie.

    Podczas wyprawy, jeszcze przed śmiercią jej fundatorów,
    zaobserwowałeś dziwne wróżby dla Hakima. Nie zdradza tego, ale
    gwiazdy mówią że znalazł się w bardzo niebezpiecznej sytuacji.

    W Adzie natomiast dziwi cię zupełny brak wróżb. Zupełnie jakby
    gwiazdy nic o niej nie wiedziały.
\subsubsection*{Publiczny opis}
\label{sec-3.4.2}

    Krótkie, położone da tyłu ciemne włosy. Ciemno niebieskie oczy
    które zdają się świecić własnym blaskiem. Kolczyki w
    brwiach. Ubiera się w granatową pelerynę i tunikę wyszywane
    srebrnymi nićmi i białą koszulę.
    Kurt jest młodym, dziewiętnastoletnim czarodziejem kolegium
    niebios. W chwili obecnej terminuje u Ady i wraz z nią wyruszył
    na wyprawę. 
    Jego ulubiony temat rozmów to\ldots{}
\subsubsection*{Cele}
\label{sec-3.4.3}

\begin{itemize}
\item kto kogo zabił
\item asystowanie Adzie
\item pomoc w bezpiecznym przeprowadzeniu wyprawy
\end{itemize}
\subsubsection*{Atuty}
\label{sec-3.4.4}

\begin{itemize}
\item mag Kolegium Niebios
\end{itemize}
\subsection*{Otto Becker}
\label{sec-3.5}

\begin{itemize}
\item kwatermistrz
\item hobbit
\item rumiane policzki, wesoły
\item nie zchodzi do grobowca.
\end{itemize}
\subsection*{Hermann Kiper}
\label{sec-3.6}

\begin{itemize}
\item sługa Malkiora
\item szczupły
\item wiek po 30
\item blady jak wampir, poważny
\item nie zchodzi do grobowca.
\end{itemize}
\section*{Postaci Tła}
\label{sec-4}

\subsection*{Wilfred Keller}
\label{sec-4.1}

   Ojciec rodu Kellerów, znanej ze swej rozrzutności rodziny z
   Talabeklandu zorganizował wyprawę po skarb 
\subsection*{Soner Sari}
\label{sec-4.2}

\subsection*{Yfez Kahraman}
\label{sec-4.3}

   Starożytny czarnoksiężnik. Władca dżinów. Drużyna wyrusza na
   wyprawę po magiczną lampę.
\subsection*{Dżin Melchior}
\label{sec-4.4}

   Zaklęty w butelce z tuszem
\subsection*{Dżin Baltazar}
\label{sec-4.5}

   Zaklęty w magicznej lampie
\subsection*{Dżin Kacper}
\label{sec-4.6}

   Dziko lata i zabija.
\section*{PLAN}
\label{sec-5}

\subsection*{Wyprawa}
\label{sec-5.1}

   Zarganizowana przez Balthazara Kellera, ojca rodu Kellerów i
   szejk Soner Sari.
\subsubsection*{Tło}
\label{sec-5.1.1}

\subsubsection*{Ginie Wilfred}
\label{sec-5.1.2}

    Zatruty. Brak widocznych ran.
\subsubsection*{Nast dnia ginie Soner Sari}
\label{sec-5.1.3}

    Znika w oazie, pozostawia po sobie amulet z wilczym kłem,
    ubrania, broń.
\subsection*{Dotarcie do Piramidy}
\label{sec-5.2}

\subsection*{Wejście do piramidy}
\label{sec-5.3}

\subsection*{Wykorzystanie klucza}
\label{sec-5.4}

\subsection*{Djinn się wydostaje, otwiera przejście dla drużyny}
\label{sec-5.5}

\subsection*{Wrota się zamykają, pułapka}
\label{sec-5.6}

\subsection*{Postaci poznają historię Baltazara}
\label{sec-5.7}

\subsubsection*{Baltazar, Kacper i Melchior niszczą miasta}
\label{sec-5.7.1}

\subsubsection*{Zostają spętani przez Yfeza}
\label{sec-5.7.2}

\subsubsection*{Yfez buduje imperium}
\label{sec-5.7.3}

\subsubsection*{Spisek czarodziejów, zabijają Yfeza, djinny:}
\label{sec-5.7.4}

\begin{description}

\item[Baltazar, potężny, zamknięty w lampie w grobowcu]\label{sec-5.7.4.1}


\end{description}
\begin{description}

\item[Melchior, mądry, zamknięty w butelce z atramentem]\label{sec-5.7.4.2}


\end{description}
\begin{description}

\item[Kacper, przewrotny, uchodzi wolno]\label{sec-5.7.4.3}


\end{description}
\subsection*{Postaci ruszają na dół / postaci ruszają na górę}
\label{sec-5.8}


\end{document}
