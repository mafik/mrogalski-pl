\documentclass[12pt,a4paper]{article}

\usepackage[utf8]{inputenc}
\usepackage{polski}
\usepackage{color}
\usepackage{anttor}
\usepackage[T1]{fontenc}
\title{Grobowiec w Chmurach}
\author{Marek Rogalski}

\begin{document}

\maketitle

\begin{quote}
\large \bf
{\huge Ś}nieżyca w Górach Krańca Świata to przeżycie, które większości dane
jest przeżyć co najwyżej raz w życiu. Zwykle zresztą jest to na samym
jego końcu. Gdy jednak przebyliście już przez samo serce zamieci,
gdzie zacinający w oczy śnieg nie pozwalał dostrzec ścieżki, którą
podążaliście, a huk wiatru zagłuszał same myśli w waszych głowach,
dotarliście do miejsca oderwanego od czasu. Zasnutej mgłą twierdzy
gdzie to nie zimno zabija, a poruszające się za mgłą cienie.
\end{quote}

\section*{Śnieżycę przetrwali}

\begin{itemize}
\item \emph{Sigmarytka} - pielgrzymka z zakonu Sigmara, podróżująca do
  Przełęczy Czarnego Ognia.
\item \emph{Przewodniczka} - jedna z przewodników wyprawy -
  zaznajomiona z górami i zdolna do walki przewodniczka.
\item \emph{Szabrownik} - kolejny z przewodników wyprawy - zna
  okolicę, lecz wiedzę tą wykorzystuje do grabienia ofiar gór.
\item \emph{Kapitan Fritz} - kapitan oddziału najemników. Do twierdzy
  trafił wcześniej niż inni.
\item \emph{Hobbit Harold} - obłąkany mieszkaniec twierdzy.
\end{itemize}

\section*{\color{red} Mistrz Gry}

\paragraph{Góra} 
Od niepamiętnych czasów po niebie przemyka latająca góra zamieszkiwana
przez ogromne stwory - choć przypominające ludzi to pozbawione twarzy,
poruszające się w niepokojący sposób, liczące sobie ponad sześć metrów
wysokości i przede wszystkim - żywiące się ludzkim mięsem.

\paragraph{Mgła}
Cały płaskowyż pokryty jest gęstą zasłoną z mgły, która nawet we
wnętrzu twierdzy uniemożliwia widok na więcej niż kilka metrów - dalej
widać jedynie cienie i kontury.

\paragraph{Lodowe Potwory}
Nazywa się je Lodowymi Potworami. Zamieszkują twierdzę na latającym
płaskowyżu, którym kierują i na którym łapią śmiertelników by przez
kolejne tygodnie upiornej podróży - żywić się nimi.

\paragraph{Lodowi Ludzie}
Lodowe Potwory łapią niektórych ludzi i dokonują na nich operacji - w
głowach umieszczają im niewielkie ziarenka - magicznie skojarzone z
większymi kulami. Kule pozwalają widzieć oczami lodowych ludzi oraz
wydawać im polecenia. W ten sposób lodowe potwory pętają niewolników,
którzy wchodzą między grupy ludzi i wabią ich na płaskowyż - by tam
stali się pożywieniem dla ich panów.

\paragraph{Twierdza}
Twierdza obejmuje całą górę - składa się z setek korytarzy długich na
wiele kilometrów. Część twierdzy znajdująca się na
powierzchni, gdzie przygodę rozpoczynają gracze jest przez nie prawie
zupełnie opuszczona - zapuszczają się tam jedynie czasami by porwać
jakiegoś śmiertelnika. Znajdują się tam opuszczone stajnie (lub coś co
wygląda na stajnie), magazyny, kuchnie i kilka pomieszczeń
mieszkalnych przystosowanych do ogromnej anatomii lodowych potworów.

\begin{description}
\item{Biblioteka} - zawiera wielkie księgi zawierające mapy świata i
  twierdzy wykonane przez lodowe potwory. Zawiera też pokaźny opis
  historii twierdzy oraz magii i technik stosowanych przez lodowe
  potwory.
\item{Sterownia} - zawiera mechanizmy linowe i parowe, za pomocą
  których można sterować lotem góry.
\item{Skarbiec} - zawiera w sobie niewyobrażalne bogactwa zebrane
  przez lodowe potwory.
\end{description}


\section*{Sigmarytka}

Jesteś jedną z Lodowych Wiedźm - spętana przez majestatyczne lecz
potworne stwory zamieszkujące latającą górę. Nazywasz ich Lodowymi
Panami, a miejsce gdzie zamieszkują - Twierdzą. Nie pamiętasz gdzie
się urodziłaś, ani jaka była twoja przeszłość - wiesz jedynie, że
Lodowi Panowie widzą twoimi oczami i kierują twoimi myślami - jesteś
od nich uzależniona, a rola, którą pełnisz w ich upiornej egzystencji
to zapewnianie im pożywienia.

Niczym zwiastun burzy - schodzisz z góry i wchodzisz między
śmiertelników by, gdy nadejdzie śnieżyca, wprowadzić ich na szczyt
góry i dalej do Twierdzy. Gdy spełnisz swoje zadanie, góra odlatuje, a
śmiertelnicy, których zwabiłaś błądzą po zamglonym labiryncie - niczym
trzoda, którą pożywią się Lodowi Panowie.

\subsection*{Cele}
\begin{itemize}
\item Nie zdradzić swojej roli w działaniach Lodowych Panów.
\item Odkryć jak Lodowi Panowie sprawują nad tobą kontrolę.
\item Wydostać się spod władzy Lodowych Panów.
\item Odkryć motywy kierujące pozostałymi postaciami.
\item (dopóki nie uwolnisz się spod władzy Lodowych Panów) Zapędzić w
  łapy Lodowych Panów jak najwięcej trzódki.
\end{itemize}

\clearpage

\section*{Przewodniczka}

Choć słyszałaś wcześniej historie o Lodowej Twierdzy - nigdy nie
sądziłaś, że mogą być prawdą. Mag Meganus, iluzjonista, który od kilku
miesięcy pracuje w Bardorfie, wynajął cię byś ruszyła wraz z jedną z
karawan i wprowadziła ją w serce śnieżycy - gdzie podobno otwiera się
przejście do innego świata zamieszkiwanego przez Lodowych Gigantów.

Zadanie jakie ci zlecił to zidentyfikowanie ludzi, którzy zostali
opętani przez Lodowych Gigantów i wykonują ich polecenia próbując
zwabić śmiertelników w łapy gigantów - masz ich przyprowadzić żywych
bądź przynieść ich głowy.

Wiesz, że \emph{za wszelką cenę} należy unikać spotkań z Lodowymi
Gigantami oraz, że przejście do świata śmiertelników otwiera się tylko
podczas śnieżycy - wtedy będziesz mogła powrócić po wykonaniu zadania.

\subsection*{Cele}
\begin{itemize}
\item Zidentyfikować i zdobyć głowy jak najwiekszej ilości opętanych.
\item Wydostać się z Lodowej Twierdzy.
\item Odkryć motywy kierujące pozostałymi postaciami.
\end{itemize}

\clearpage

\section*{Szabrownik}

Zebrałeś już dość informacji by dokonać największego skoku w twoim
życiu. Od miesięcy szukałeś i przypytywałes ludzi, którzy podobno
trafili do legendarnej Lodowej Twierdzy - miejsca ukrytego gdzieś w
górarch, do którego czasami trafiają nieopatrzni podróżnicy.

Szukałeś wzorów, wypytywałeś historyków i innych badaczy - czujesz, że
możesz to zrobić, ale musisz to zrobić sam. Lodowi Giganci - którzy
zamieszkują Twierdzę i zabijają wszystkich intruzów zebrali w niej
niemały skarb, który zamierzasz wykraść.

\subsection*{Cele}
\begin{itemize}
\item Spenetrować Lodową Twierdzę i zidentyfikować najbardziej
  opłacalne cele.
\item Wykraść przedmioty o jak największej wartości.
\item Wydostać się z Lodowej Twierdzy.
\item Odkryć motywy kierujące pozostałymi postaciami.
\end{itemize}

\clearpage

\section*{Kapitan Fritz}

Trafiłeś tu wiele dni temu, wtedy jeszcze razem ze swoim
pięcioosobowym oddziałem. Przez wiele dni podróżowałeś uliczkami
zamglonej budowli. Początkowo było was sześcioro - pełen oddział
ochroniarzy. Po kilku dniach ciężkiej głodówki i przemykania
korytarzami zdecydowaliście się zrobić to co zrobić
musieliście. Zabiliście najsłabszego. Zjedliście go.

To nie był jednak koniec. Czas mijał dalej - ogromne cienie oraz
krzyki ludzi w oddalonych częściach Twierdzy doprowadzały was do
szaleństwa jednak to głód doprowadził was do największych zbrodni.

Pozostałeś tylko ty i twój najlepszy przyjaciel - także oficer, choć
gdy zjedliście własny oddział stopnie przestały się już liczyć. W
ostatnim akcie desperacji przyczailiście się na jednego z olbrzymów i
postanowiliście go zaatakować.

Ty przetrwałeś jedynie z kilkoma siniakami. Twój towarzysz nie miał
tyle szczęścia. Choć trwało to ułamek sekundy, w twojej głowie
pozostało na długie godziny. Olbrzym o czerwonej skórze, pozbawiony
oczu nosa czy uszu - postać bez twarzy - złapał twojego przyjaciela,
podnosząc zmiażdżył mu żebra, a gdy ten nie przestawał się szamotać
wyrwał mu rękę i odszedł niosąc to wyjące, okaleczone ciało, które
niegdyś było twoim przyjacielem.

Od wielu godzin siedziałeś ukryty w magazynie, za belami materiału, w
pobliżu waszej kryjówki. Jedyne co słyszałeś przez długie godziny to
echo, pogłos krzyku twojego towarzysza, którego nie mogłeś wypędzić z
głowy, aż usłyszałeś głosy innych ludzi.

\subsection*{Cele}
\begin{itemize}
\item Nie dopuść by ktoś dowiedział się że jesteś kanibalem.
\item Dowiedz się co stało się z twoim przyjacielem.
\item Wydostań się z Twierdzy.
\item Odkryć motywy kierujące pozostałymi postaciami.
\end{itemize}

\clearpage

\section*{Hobbit Harold}

\paragraph{ \color{red} Dla Mistrza Gry}
W razie gdy któraś z postaci jest zbyt bliska sukcesu, chcesz
wydłużyć, bądź skrócić sesję lub któraś z postaci zginęła (a warto
zabić postać któregoś z graczy - tylko po to żeby przestraszyć
pozostałych), hobbita Harolda możesz wprowadzić jako jeden z wielu
archetypów:
\begin{itemize}
\item Lodowego Człowieka
\item \emph{Zwierzątko} Lodowych Potworów
\item Pewnego siebie łowcy
\item \emph{Szczura}, kóry spenetrował wszystkie zakątki twierdzy
\item Rozdwojenie jaźni - dwa z powyższych
\end{itemize}


\end{document}
